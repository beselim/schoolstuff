\documentclass{article}

\usepackage{amsmath}
\usepackage{graphicx}
\usepackage{cleveref}

\begin{document}

\section{Variatonal methods}
\subsection{Why use weighted-integral statements?}
We typically seek to approximate quantities with solutions of the form:
\begin{equation}
u(x) \approx U_N(x) = \sum_{j+1}^N c_j \phi_j (x) + \phi_0 (x)
\end{equation}
where we \textit{choose} $\phi_j$. Consequently, $U_N$ is known if $c_j$ are known.

However, requiring $U_N$ to satisfy the differential equation at any point in the domain
is not generally possible, otherwise we would've found the exact solution. Moreover, we
only have one equation to work with, whereas we have $j$ constants to determine.

On the other hand, we can require $U_N$ to satisfy the equation in a weighted-integral sense:
$$
\int_\Omega w(x) R dx = 0
\label{eq:weighted-integral}
$$
We can now obtain as many linearly independent equations as there are linearly independent
functions for $w(x)$.

Note that the exact analytical solution will satisfy the above equation for any choice of
$w$.
\subsection{Why Weak Form?}
\label{sec:weakform}
Suppose \Cref{eq:weighted-integral} looks like this:
$$
0 = \int_\Omega w(x) \left[ -\frac{d}{dx} \left(a \frac{du}{dx} \right) - f \right] dx
$$
This is equivalent to the differential equation but does not include any boundary conditions.
Moreover, our function $u$ needs to be twice differentiable.

Let's look at the weak form:
$$
0 = \int_\Omega \left( a \frac{dw}{dx}\frac{du}{dx} - wf \right) dx
- \underbrace{\left[ wa \frac{du}{dx} \right]^L_0}_\text{Boundary Condition}
$$

Now, the weak form reduces the differentiability requirement on $u$. In addition, the natural
boundary conditions are included in the weak form, thus our choice of $U_N$ must only satisfy
the essential boundary conditions!

The weak form is obtained in three steps (starting from the differential equation)
\begin{enumerate}
    \item Obtain weighted-integral form.
    \item Integrate by parts to distribute differentiation between $w$ and $u$.
    \item Apply required BC to the boundary term.
\end{enumerate}
Finally, the principle of minimum total potential energy and the weak form are equivalent
mathematically (apparently).

\subsection{Ritz vs. Weighted-Residual}
The Ritz method is \textbf{NOT} a special case of the weighted residual method. Ritz uses
the weak form.

Functions $U_N$ chosen in weighted-residual methods must satisfy the natural and essential BCs,
whereas Ritz only needs to satisfy the natural BCs since essential BCs are included in the
weak form.

It should be noted that one can formulate an FEM problem using any method, whether it be
least-squares, Ritz (also known as weak-form Galerkin).

\subsection{Why do we need to take the variation?}
We really shouldn't worry about this anymore. You can arrive at the weak form using
the principle of minimum total potential energy -- which uses variations -- or from the
differential equation using the steps in~\Cref{sec:weakform}

\subsection{Why FEM?}
Chosen functions $U_N$ are required to satisfy BCs of the problem. This is fine for simple
geometries or in 1-D, but is near impossible for complex geometries. FEM allows us to split
the domain into multiple simple domains and satisfy the equations over each domain. Boundary
conditions for the divided elements come from the problem's BCs and the continuity requirement.
\end{document}
