\section{Question 2}
The bending moment $M$ and transverse deflection $w$ in a beam according to the
Euler-Bernoulli beam theory are related by:
\begin{equation}
    \label{eq:governbeam}
    -EI\diff{^2 w}{x^2} = M(x)
\end{equation}
In this case, a linearly distributed load $q(x)$ is applied over the whole length
$L$ of the beam:
\begin{equation}
    q(x) = q_0 \left(1 - \frac{x}{L}\right)
\end{equation}

\subsection{Analytical Solution}
The exact solution is obtained by first finding an expression for $M(x)$ by
taking a cut along the beam:
\newcommand{\pfrac}[2]{\ensuremath{ \left( \frac{#1}{#2} \right) }}
$$
    M(x) = -\frac{q_0 L^2}{6}\left[
        1 - 3\pfrac{x}{L} + 3\pfrac{x^2}{L^2} - \pfrac{x}{L}^3
    \right]
$$
Integrating~\Cref{eq:governbeam} once gives the deflection angle $\theta$:
$$
    \theta(x) = -\frac{q_0}{E I} \left[
        -\pfrac{x^4}{24L} + \pfrac{x^3}{6} - \pfrac{L x^2}{4} + \pfrac{L^2 x}{6} + C_1
    \right]
$$
Applying the boundary condition $\theta(0) = 0$ yields $C_1 = 0$.

Integrating once more gives the deflection $w$:
$$
    w(x) = -\frac{q_0}{EI} \left[
        -\pfrac{x^5}{120 L} + \pfrac{x^4}{24} - \pfrac{L x^3}{12} + \pfrac{L^2 x^2}{12}
        + C_2
    \right]
$$
and applying $w(0) = 0$ yields $C_2 = 0$.

Thus, the maximum displacement is at $L$ for $q_0 = EI = L = 2$ is given by:
$$
    w_{max}(x) = w(L) = \frac{2^4}{30} = \frac{8}{15} \approx 0.5333
$$

\subsection{FEM solution}
As we learned in class, the weak form -- if we use only one element and
set $c_f = 0$ -- is given by:
\begin{equation}
    \label{eq:weakbeam}
    0 = \int_0^L v\left(
        EI \diff{^2w}{x^2}\diff{^2 v}{x^2} - vq
    \right) \text{d}x
    + \left[
        v \diff{}{x}(EI\diff{^2 w}{x^2}) - \diff{v}{x}EI\diff{^2w}{x^2}
    \right]^L_0
\end{equation}
We use the following approximation on $w$:
\begin{equation}
    w_h^e(x) = \sum_{j+1}^4 \Delta_j^e \phi_j^e(x)
\end{equation}
where the interpolation functions are given in the notes. This leads to the following
element -- and global in this case -- stiffness matrix:
\begin{equation}
    [K] = \frac{2 E I}{L^3} \left[ \begin{array}{cccc}
        6 & -3L & -6 & -3L\\
        -3L & 2L & 3L & L^2\\
        -6 & 3L & 6 & 3L\\
        -3L & L^2 & 3L & 2L^2
    \end{array} \right]
\end{equation}
Now, in order to obtain the source term we need to carry out the following integration for
each $v$:
$$
    \int_0^L \phi_i(x) q(x) \text{d}x
$$
which yields:
\begin{equation}
    [Q] = \frac{q_0 L}{12} \left \{ \begin{array}{c}
        6 \\ -L \\ 6 \\ L\\
    \end{array} \right \} + \frac{q_0}{60} \left \{ \begin{array}{c}
        -9L \\ 2L^2 \\ -21L \\ -3L^3 \\
    \end{array} \right \}
\end{equation}
Finally, applying boundary conditions at the fixed end ($w(0) = \theta(0) = 0$) allows
us to remove the first and second rows/columns and solve for the remaining degrees
of freedom. The resulting displacement $w_1^2$ is equal to 0.5333.

Thus, the exact solution of $w$ is found for $x = L$.
