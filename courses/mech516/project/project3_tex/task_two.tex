%!TEX root = main.tex

\section{Numerical simulation of a practical problem: blast wave interaction with an obstacle (wall).}

\begin{quote}
\em \centering The VanLeer limiter was chosen as a result of a compromise between accuracy and potential oscillations, which are bound to occur in this Task.
\end{quote}
It should also be noted that the Courant number was fixed at 0.8 and that 6 ghost control volumes were used (three on either side).

\subsubsection{Grid Information}
A grid was constructed from $x=0.001$ to $x=1.501$. Fifteen (15) nodes were included in the initial pressure region. The end time was set to 0.5.

\subsection{Pressure and Density Distribution}

A video was captured and can be found at \url{https://youtu.be/8LoSgmQwReQ}. The blast wave is very obvious: the propagating shock hits the wall and bounces back but creating a large pressure wave. The pressure distribution makes sense, as the pressure spikes when the shock hits the wall and bounces off, leaving a high pressure region behind it. The density distribution is acting weird at the last node and I'm not sure why.

The simulation shown in the video uses only 151 nodes.

\subsection{Pressure History}
Pressure history was queried for $x=1.351$. It was ensured that this point always be a part of the grid! This proved to be problematic at first; one must exercise care when picking a point to query and then modifying the grid. The number of nodes $N$, excluding ghost cells, varied from 51 to 1001 with increments of 50. The results are plotted in~\Cref{fig:task2_peaks}.

The relative difference in maximum pressure peak between consecutive simulations was also calculated 1\%. The relative difference was calculated as:
\begin{equation}
    \epsilon_i = \dfrac{\big| \hat{P_i} - \hat{P_{i-1}} \big|}{\dfrac{\hat{P_i} + \hat{P_{i-1}}}{2}}
\end{equation}
where $\hat{P_i}$ is the maximum pressure for a given number of nodes.

It was found that $\epsilon(1000)$, the relative change for 1000 when compared to 950, was under 5\%. This could be considered low enough. The maximum peak pressure for this mesh is 91.

It is interesting to note that the strength of the blast at the wall diminishes as the mesh is refined. One could also have performed a study on increasing the number of ghost cells -- this was not done here.
%
\begin{figure}
    \centering
    \includegraphics[width=0.8\textwidth]{./figs/task2_peaks}
    \caption{Evolution of highest peak with mesh refinement}\label{fig:task2_peaks}
\end{figure}
%

