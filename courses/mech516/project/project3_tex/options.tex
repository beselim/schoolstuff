%!TEX root = main.tex

%%%%%%%%%%%%%%%%%%%%%%%%%%%%%%%
% Additional Packages/Options %
%%%%%%%%%%%%%%%%%%%%%%%%%%%%%%%
% \setlist{nosep}
% \hypersetup{hidelinks}
\usepackage{pdflscape}
\usepackage{titlesec}
\usepackage{abstract}
\usepackage{listings}

%%%%%%%%%%%%%%%%%
% Title Options %
%%%%%%%%%%%%%%%%%
\usepackage{mypage}
\school{McGill University}
\course{Computational Gasdynamics}
\coursenum{MECH 516}
%Add \\[0.3cm] for new line.
\title{Mini-Project 3}
\student{Selim \textsc{Belhaouane}}
\studentnum{260450544}
\date{\today}

%%%%%%%%%%%%%%%%%%%%%%%%%
% Additional Formatting %
%%%%%%%%%%%%%%%%%%%%%%%%%
%Horizontal line below section.
\sectionfont{\sectionrule{0pt}{0pt}{-5pt}{0.8pt}}
%Section numbering depth. Value of 2 means numbering ends with subsections.
\setcounter{secnumdepth}{2}
%Table of contents section depth. Same as above.
\setcounter{tocdepth}{2}
% \numberwithin{equation}{subsection}
% \numberwithin{figure}{subsection}
\titleformat{\section}[block]{\normalfont\large\bfseries}{Task \thesection.}{0.5em}{}
\titleformat{\subsection}[block]{\normalfont\large}{Exercise \thesubsection.}{0.5em}{}
\renewcommand{\abstractname}{}