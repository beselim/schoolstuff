%Options : [NoTitlePage][Custom][Bib][Nomencl]
\documentclass{SelimArticle}
%!TEX root = main.tex

%%%%%%%%%%%%%%%%%%%%%%%%%%%%%%%
% Additional Packages/Options %
%%%%%%%%%%%%%%%%%%%%%%%%%%%%%%%
% \setlist{nosep}
\hypersetup{hidelinks}

%%%%%%%%%%%%%%%%%
% Title Options %
%%%%%%%%%%%%%%%%%
\usepackage{mypage}
\school{McGill University}
\course{Computational Aerodynamics}
\coursenum{MECH 539}
%Add \\[0.3cm] for new line.
\title{Project 5}
\student{Selim \textsc{Belhaouane}}
\studentnum{260450544}
\date{\today}

%%%%%%%%%%%%%%%%%%%%%%%%%
% Additional Formatting %
%%%%%%%%%%%%%%%%%%%%%%%%%
%Horizontal line below section.
\sectionfont{\sectionrule{0pt}{0pt}{-5pt}{0.8pt}}
%Section numbering depth. Value of 2 means numbering ends with subsections.
\setcounter{secnumdepth}{3}
%Table of contents section depth. Same as above.
\setcounter{tocdepth}{2}
\numberwithin{equation}{section}
\numberwithin{figure}{section}

\newcommand{\ra}[1]{\renewcommand{\arraystretch}{#1}}
\begin{document}
\mytitlepage
\section{Code}
Code is hosted on \href{https://github.com/Kreger51/mech_516/tree/master/project/gasdynamics}{GitHub}\footnote{Selim Belhaouane, MECH 516 Project, (2015), GitHub Repository, \url{https://github.com/Kreger51/mech_516/tree/master/project/gasdynamics}}. GitHub automatically provides syntax-highlighting, which makes code easier to read. 

\section{Report}
All plots can be found on the next page. For the first plot of each Problem, an x-t diagram with the associated waves is shown. The legend is as follows:
\begin{description}[nolistsep]
    \item [Contact Surface:] Black dashed line.
    \item [Shockwave:] Red solid line.
    \item [Rarefaction Wave:] Five blue solid lines. The outer/inner lines represent the tail and the head. 
    \item [The green horizontal dashed line] shows at what time results are sampled.
\end{description}  
Moreover, the intersections with waves are shown and span all graphs. 

Regarding left-facing and right-facing waves, one needs only look at where the contact surface is: if the wave is to the left of the contact surface, the wave is left-facing and vice-versa. For conciseness, here is the breakdown of waves for each problem:
\begin{description}[nolistsep]
    \item [Problem 1:] Left-facing expansion fan and right-facing shock.
    \item [Problem 2:] Left-facing AND right-facing expansion fan.
    \item [Problem 3:] Left-facing expansion fan and right-facing shock.
\end{description}
\newpage
\includepdf{figs./plots}

% \bibliographystyle{IEEEtran}
% \bibliographystyle{plain}
% \bibliography{dabib}
\end{document}