\section{Aerodynamic performance}
Aerodynamic coefficients are tabulated in~\Cref{tab:aero}. Results were compared
to a 1988 paper by Ladson~\cite{ladson1988effects}. This paper presents experimental data of
the low-speed aerodynamic characteristics of the NACA 0012 airfoil, which is
in line with this project.

The rest of this section will be split between an analysis of the MECH 539 results alone and
a comparison between MECH 539 and experimental data.


\subsection{MECH 539 results}
Let's look at lift first. Most textbooks will explain lift as a difference in pressure
between the upper and bottom surface, which results in a net upwards force. In other words,
there is usually no mention of viscous forces when considering lift. Thus, it makes sense
that, according to~\Cref{tab:aero}, $c_{l_p}$ is essentially the sole contributor to
lift.

Now, the same cannot be said of drag, where $c_{d_f}$ and $c_{d_p}$ contribute almost
equally to the total drag. Additionally, it can be said that viscous forces
primarily contribute to drag, while pressure forces have a large relative impact on both
lift and drag.

Finally, it can be seen that $c_l$ is around 2 orders of magnitude higher than $c_d$, i.e.
$l/d \approx 100$.
\subsection{Comparison with experimental}
Experimental data in~\Cref{tab:aero} was obtained from Table I
in~\cite{ladson1988effects}. The latter table lists values of lift and drag coefficients
-- but not the contributions from viscous and pressure forces -- at a Reynolds number of 2
million and Mach number of 0.15 for various angles of attack. While the Mach number
does not correspond to that of the provided MECH 539 results (0.1) both are low enough
that the flow is essentially incompressible, making the Reynolds number a more important
value to match -- and these match exactly. In other words, changes in Mach number
at subsonic speeds should not be expected to significantly affect aerdoynamic performance
as long as the Reynolds number is kept constant.  Finally, the experimental
data needed to be interpolated in order to obtain values at $\alpha = 8.0$.

The MECH 539 and Ladson results are of the same orders of magnitude but differ by around
3\% and 20\% for lift and drag respectively. Of course, sources of discrepancy in
the experimental data are always expected, especially in the presence of turbulent flow
since it is inherently unsteady and very sensitive to inputs~\cite{pope}.

Moreover, as will be discussed in~\Cref{sec:bl}, the numerical simulation predicts
three zones of recirculation on the airfoil. Since a RANS solver was used, it is expected
that values should be slightly off, as was discussed in class.
\begin{table}
    \centering
    \caption{Aerodynamic coefficients of provided data and experimental data from
        ~\cite{ladson1988effects}.}
    \label{tab:aero}
    
\begin{tabular}{@{} l ccc c ccc}
\toprule
 & \multicolumn{3}{c}{Lift} & \phantom{a} & \multicolumn{3}{c}{Drag}\\
\cmidrule{2-4}\cmidrule{6-8}
Source & Pressure & Viscous & Total && Pressure & Viscous & Total\\
\midrule
MECH 539 & 0.8091 & -0.0001 & 0.8089 && 0.0048 & 0.0049 & 0.0098\\
Ladson & - & - & 0.8386 && - & - & 0.0079\\
\bottomrule
\end{tabular}

\end{table}
