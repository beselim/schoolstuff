\section{Momentum deficit}
The velocity profiles are shown in~\Cref{fig:mom}. It should be noted that the normalized
absolute velocity $U/U_\infty$ was plotted, where:
\begin{align*}
    U &= \sqrt{u^2 + v^2}\\
    U_\infty &= U\big|_{x=-10, y=0}
\end{align*}
It should also be noted that interpolation was used in order to get the profile at constant
$x$, as opposed to plotting a constant $i$ grid line.

Several observations can be made from the plot:
\begin{enumerate}
    \item There is indeed a momentum deficit, i.e. a region of lower velocity past the airfoil. This
        momentum deficit can be explained by the convection of the boundary layer on the airfoil --
        recall that momentum in the boundary layer is reduced compared to that of the freestream.
    \item The momentum deficit is gradually \textit{diffused} as it
        is convected. This is done through viscosity,
        i.e. the diffusion term in the Navier-Stokes equations.
    \item The momentum deficit in the wake is directly related to the total drag felt by the
        airfoil. This can be shown using conservation of momentum around the airfoil (common
        question in a Fluid Mechanics class). In fact, measuring this deficit is a common
        way of experimentally determining drag~\cite{genc}.
    \item The bump moves upwards as $x$ increases. This is merely due to the angle of attack.
        If we were to plot the profile perpendicular to the flow, this would probably not
        occur.
\end{enumerate}
\begin{figure}
    \centering
    \includegraphics[width=1.0\textwidth]{./figs/mom}
    \caption{Speed profile at various locations.}\label{fig:mom}
\end{figure}

