\section{Number of panels}

The panel method essentially consists in discretizing an integral with a finite linear combination of
simple solutions, in this case doublets with a constant strength over the length of each panel.
Thus, it is expected that increasing the number of terms in the sum would increase the
accuracy of the discrete solution relative to the analytical one. Moreover, panels all have
equal lengths in PABLO -- this is important.

The pressure distributions are shown in~\Cref{fig:q1}. Most of the discrepancy in pressure
results is found at the airfoil's extremities, specifically the leading edge and trailing edge.
This can be explained by two factors:
\begin{enumerate}
    \item The increased curvature of the airfoil geometry at the leading edge, which requires a higher
        number of panels to be modelled correctly by flat panels.
    \item The increased curvature of the pressure distribution at the leading and trailing edge, which
        again requires a higher number of panels since panels are of constant strength.
\end{enumerate}
Unfortunately, determining the minimum necessary number of panels to acquire a reasonably accurate
solution would require eyeballing.

Lift and drag coefficients are tabulated in~\Cref{tab:q1}. From these results, one may say that using as
little as 60 panels is sufficient if the user is fine with a $\Delta C_L$ and $\Delta C_D$ below 1\%. $\Delta C_L$ as
a function of the number of panels NP is calculated as follows:
\begin{equation*}
    \Delta C_L(NP) = \frac{|C_L(NP) - C_L(100)|}{C_L(100)}
\end{equation*}
Of course, this is dependent on the geometry and flow parameters.
\begin{table}
    \centering
    \caption{Effect of number of panels (NP) on lift and drag coefficients. $\Delta C_L$ and $\Delta C_D$
        are calculated with respect to the values obtained for NP = 100.}
    \label{tab:q1}
    \input{tableq1}
\end{table}

\begin{figure}[H]
    \centering
    \includegraphics[width=1.0\textwidth]{./figs/q1.pdf}
    \caption{Effect of number of panels (NP) on pressure distribution over the airfoil and at
        select locations.}\label{fig:q1}
\end{figure}

