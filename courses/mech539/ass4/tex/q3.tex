\section{Aerodynamic characteristics at various angles}
\subsection{Pressure coefficient}
The $c_P$ distributions are shown in~\Cref{fig:q3_cp}. It can be seen that the difference
in pressure between the top and bottom surface increases as the angle of attack increases. Incrementing
the angle of attack has the effect of decreasing $c_P$ everywhere on the top surface and increasing
$c_P$ everywhere on the bottom surface.

One could also look at it in terms of velocity: increasing the angle of attack also has a positive
effect on the acceleration of the fluid at the leading edge. In other words, the flow turns more
more agressively.

\begin{figure}
    \centering
    \includegraphics[width=0.8\textwidth]{./figs/q3_cp}
    \caption{Effect of angle of attack on pressure distribution. The angles of attack vary from 0 to
        10 in increments of 2.}
    \label{fig:q3_cp}
\end{figure}

\subsection{Lift and drag}
The lift curve slope and drag polar are shown in~\Cref{fig:q3_lift}. From the previous
section on the differences in pressure coefficient, it is expected that increasing the
angle of attack also increase lift, since lift is generated from the difference in pressure
between the top and bottom surface. Moreover, it can be seen that there is a
perfectly linear fit between the angle of attack and the lift coefficient: increasing
the angle of attack by 1 degree increases the lift coefficient by 0.116.

The drag polar reveals a particularly interesting result: increasing the lift coefficient becomes
more and more costly in terms of drag coefficient. In other words, the ratio of lift over drag
decreases as the lift coefficient, and thus angle of attack, increases. I believe the only source
of drag in these simulations is \textit{pressure drag}: the horizontal
component of the integral of pressure over the airfoil surface. Geometrically speaking, it is expected
that increasing the angle of attack would also ``rotate'' the pressure integral vector, i.e. increase
the contribution from the horizontal component.
\begin{figure}
    \centering
    \includegraphics[width=\textwidth]{./figs/q3_lift}
    \caption{Lift curve slope and drag polar.}\label{fig:q3_lift}
\end{figure}


