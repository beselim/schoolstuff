
\lstset{language=[90]Fortran,
  basicstyle=\footnotesize\ttfamily,
  lineskip=-2pt,
  frame=single,
  keywordstyle=\color{blue},
  commentstyle=\color{PineGreen},
  numbers=left,
  morecomment=[l]{!\ }% Comment only with space after !
}
The listing below shows the meat of the code.
\texttt{implicit none} is used,
which means all undeclared variables that appear in the listing are globals.
These globals are all constants.
\begin{lstlisting}[caption={Relevant part of the linear solver.
\texttt{solverID} is a global integer variable that has the mapping:
1=Jacobi, 2=Gauss, 3=SOR},
label=lst:fortran]
subroutine update(u, maxerr)
    real(sp), dimension(:), intent(inout) :: u
    real(sp), intent(out) :: maxerr
    real(sp), allocatable, dimension(:) :: uold
    real(sp) :: s, corr, err
    integer, dimension(4) :: stencil
    integer :: i, j, row
    maxerr = 0
    if (solverID .eq. 1) then
        allocate(uold(size(u)))
        uold = u
    end if
    do i = 2, nx-1
        do j = 2, nx-1
            row = get_row(i, j)
            ! Calculate correction
            stencil = (/ row-nx, row-1, row+1, row+nx /)
            if (solverID .eq. 1) then
                s = sum(uold(stencil))
            else
                s = sum(u(stencil))
            end if
            corr = (BI - AIJ*s)/AII
            if (solverID .eq. 3) then
                corr = (1.0 - relax)*u(row) + relax*corr
            end if
            ! Calculate error
            err = abs(corr - u(row))
            maxerr = max(maxerr, err)
            ! Assign correction
            u(row) = corr
        end do
    end do
end subroutine
\end{lstlisting}
