\section{NACA airfoil investigations}

\subsection{Effect of thickness}
The effects of thickness on aerodynamic performance is shown in
~\Cref{fig:q5_thickness}. Since there is no camber, the pressure distribution
on the top and bottom surfaces is identical. Moreover, increasing the thickness
also increases the drag since the airfoil becomes more and more blunt and
wall-like. Finally, transition location changes only slightly with thickness variation.
\begin{figure}[H]
    \centering
    \begin{subfigure}{0.8\textwidth}
    \includegraphics[width=\textwidth]{./figs/q5_thickness_cp}
    \end{subfigure}
    \\
    \begin{subfigure}{0.8\textwidth}
    \includegraphics[width=\textwidth]{./figs/q5_thickness_polar}
    \end{subfigure}
    \caption{Effect of thickness on aerodynamics}\label{fig:q5_thickness}
\end{figure}

\subsection{Effect of camber}
The effects of camber on aerodynamic performance is shown in
~\Cref{fig:q5_camber}. Increasing camber has a very noticeable effect on the
pressure difference between top and bottom surfaces; pressure at the top is decreased
and pressure at the bottom is increased as the camber increases. This is
usually desirable. The transition location on the bottom surface is pushed toward
the leading edge as the camber is increased, while the transition on the upper surface is pushed
back toward the trailing edge for camber values less or equal to 6.

Increasing the camber past 6 has a dramatic effect on drag, as seen in the drag
plot -- the drag slowly increases with camber prior to that. This can be attributed to the
fact that \textbf{laminar separation}, as opposed to turbulent transition, occurs for camber values higher
than 6, as shown in the $C_P$ plot. Separation is generally to be avoided for this exact reason,
i.e. it increases the drag significantly.
\begin{figure}[H]
    \centering
    \begin{subfigure}{0.8\textwidth}
    \includegraphics[width=\textwidth]{./figs/q5_camber_cp}
    \end{subfigure}
    \\
    \begin{subfigure}{0.8\textwidth}
    \includegraphics[width=\textwidth]{./figs/q5_camber_polar}
    \end{subfigure}
    \caption{Effect of camber on aerodynamics}\label{fig:q5_camber}
\end{figure}

\subsection{Effect of camber location}
The effects of camber location on aerodynamic performance is shown in
~\Cref{fig:q5_camberloc}. Laminar separation on the top surface only
occurs for camber locations 2 and 3, and only for 2 on the bottom surface.
This obviously leads to increased drag for those values of camber location.

Past camber locations of 3 and until 5, transition location on the
upper surface is delayed, i.e.
pushed back toward the trailing edge. After camber location values of 5, the
transition location moves back toward the leading edge again. This explains why the drag
is lowest for a camber location value of 5.
\begin{figure}[H]
    \centering
    \begin{subfigure}{0.8\textwidth}
    \includegraphics[width=\textwidth]{./figs/q5_camberloc_cp}
    \end{subfigure}
    \\
    \begin{subfigure}{0.8\textwidth}
    \includegraphics[width=\textwidth]{./figs/q5_camberloc_polar}
    \end{subfigure}
    \caption{Effect of camber location on aerodynamics}\label{fig:q5_camberloc}
\end{figure}
