\title{Low-Reynolds-Number Airfoils}
\author{}
\date{}
\maketitle
This summary is divided into sections corresponding to the sections in the paper.
\section{Introduction}
\label{sec:intro}

\subsection{Background}
\label{sub:Background}
Airfoil design is a crucial part of any aeronautical application due to its contribution to
both lift and drag. The optimal airfoil shape largely depends on the application, more
precisely the size and speed of the wing as well as the fluid properties. Due to the
relatively large number of factors, it is convenient to use a dimensionless quantity to
characterize the flow: this quantity is precisely the Reynolds number.

The Reynolds number is defined as $\mathrm{Re} = Vc/\nu$, where $V$ is the flight speed, $c$ is the
chord and $\nu$ is the fluid's kinematic viscosity. Thus, instead of referring to specific
applications, e.g. a model airplane with a cruising speed of 10 m/s at 15,000 ft. altitude,
we can simply state that the Reynolds is around $10^5$. Moreover, this allows the engineer
to perform experiments with reduced-size models -- all he has to do is ensure that the
dimensionless quantity is the same. Finally, the Reynolds number is used to characterize different
flow regimes, namely laminar and turbulent flow; the larger the Reynolds number, the more turbulent
the flow is.

\subsection{Motivation}
\label{sub:Motivation}
It is well known that airfoils perform better in the turbulent regime due to the resistance to flow
separation, this will be discussed further in~\Cref{sec:fluidmechanics}.
At the time the article was written, most of the research effort went into designing airfoils for the
latter regime and some fairly impressive results were obtained. In other words, the design of airfoils
for high Reynolds number (above $10^6$) was not a challenge anymore.

However, developments in small remotely piloted vehicles (RPVs) was on the rise, and these vehicles
need to perform at much lower Reynolds numbers -- hence the article's title. More precisely, low Reynolds
numbers are taken to be in the range of $10^4 - 10^6$.

This begs the question. How does one measure an airfoil's performance? A convenient parameter is the
lift-to-drag ratio $C_L/C_D$, where higher values are desirable.

\section{Fluid Mechanics}
\label{sec:fluidmechanics}
This section aims to provide a summary of the physics governing flow around an airfoil,
namely flow separation, the different characteristics of turbulent and laminar flow,
flow regime transitions, reattachment and stall.
A thorough analysis of each of these is beyond the scope of this text, consequently
only the key points are underlines.

\subsection{Flow Separation}
Separation is typically due to fluid velocity near the wall boundary switching directions, which creates
a circulation bubble.  All airfoils have regions of lower-than-static pressure, usually on the lifting surface.
Flow in that region must inevitably return to free-stream conditions at the trailing edge, which leads
to an adverse pressure gradient. Since velocity and momentum are diminished near the wall, it may occur
that the fluid in that region is reduced up to the point where it starts flowing in the opposite direction,
which results in recirculation in said region.

\subsection{Turbulent vs. Laminar}
It was mentioned above that turbulent flow is more resistant to flow separation. This is due to one of the main
differences between the two regime: the velocity gradient perpendicular to the wall direction. This gradient is
much higher in turbulent flow. Thus, it is harder for the latter flow to undergo a velocity reversal. Again,
the performance of low-Reynolds-number airfoils is entirely due to this distinction.

\subsection{Flow Regime Transition}
Provided there is no separation, flow over a solid boundary always increases in Reynolds number. Thus, it is
possible for initially laminar flow to undergo a transition to turbulent flow. The location of this transition
is of great interest to airfoil designers because of the reasons mentioned above. Specifically, one would
want transition to occur upstream of severe adverse pressure gradients.

\subsection{Reattachment}
After laminar separation, the flow may transition to being turbulent, in which case reattachment of the
boundary layer can occur. This forms a separation bubble. The distance from separation to reattachment
is dictated by the Reynolds number. In some cases, the airfoil is too short for reattachment to occur. The higher
the Reynolds number, the less distance is required to reattach.

\subsection{Stall}
Stall is a phenomenon that occurs when the separation bubble extends so far that reattachment no longer occurs -- this
is also referred to as a separation bubble burst. This typically happens due to an overly high angle of attack. Stall has
severe consequences on performance and should be avoided at all costs.

\section{Experimental Testing of Airfoils}
Airfoil testing in low Reynolds number conditions is difficult for the following key reasons:
\begin{itemize}[noitemsep, nolistsep]
    \item While we would like to perform a two-dimensional analysis, flow is inherently three-dimensional.
    \item Small changes can trigger large effects in the regions of interest: stall and separation.
    \item Wall effects pose problems in wind-tunnel testing.
    \item In free-flight testing, it is a challenge to isolate airfoil performance from the rest of the glider.
    \item The model shape may not be true.
\end{itemize}
Consequently, it is frequent for tests in different facilities yield different results for the same airfoil shape.

\section{Theoretical Design of Airfoils}
Analytic methods involving conformal transformations as well as numerical computations can be used
to reliably predict airfoil performance, provided no separation occurs. There are two main techniques to
design an airfoil. The first is referred to as \emph{direct procedure}, where for a given airfoil shape
the pressure field is determined. The second is referred to as \emph{inverse procedure} and consists of
determining the airfoil shape given a required pressure field. The latter method has proved to be
more successful.

However, the successful designs have mostly been for high Reynolds number airfoils. In other words, there was still
work to be done when it came to low Reynolds number at the time the article was written.

\section{Special-Purpose Airfoils}
In this section, the author highlights some applications requiring peculiar airfoil shapes. In the first case, a
flap was needed to ensure acceptable performance in all regimes encountered by the airfoil. In the second, the constraint
was purely geometrical: most of the upper part of the airfoil had to be completely flat; airfoil performance
was improved by tweaking the nose and undersurface.

\section{Parting Comments}
The author explains how knowledge in the fluid mechanics field has improved, which should hopefully lead to
better designs in the future. I believe his hopes came true.
\label{sec:Parting Comments}

